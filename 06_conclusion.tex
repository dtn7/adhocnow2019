\section{Conclusion}
\label{sec:conclusion}
We presented an open source DTN implementation, called \dtn, of the recently released Bundle Protocol BP7 (draft version 13), written in the Go programming language.
\dtn is designed to offer extensibility and supports multiple use cases, such as enabling communication in emergency and disaster scenarios or providing connectivity for rural areas.
Furthermore, we presented results of a comparative experimental evaluation of \dtn and other DTN systems including Serval, IBR-DTN, and Forban.
Our results indicated that \dtn is a flexible and efficient open-source multi-platform implementation of the most recent version of BP7.

%With \dtn, we have laid the foundation for future research in the field of DTN.
There are several areas for future work.
For example, the BP does not define any kind of security or privacy mechanisms, although optional extension exist.
This opens the field of DTN-related security and privacy research based on \dtn.
Furthermore, for sensor networks or deployments in rural areas, \dtn's energy consumption should be evaluated.
Due to \dtn's modular routing interface, new DTN routing algorithms for vehicular ad-hoc networks or UAV-based information dissemination should be investigated.
Finally, new Convergence Layers based on emerging radio technologies, such as LoRa or mmWave communication, could be developed.
